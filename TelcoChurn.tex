% Options for packages loaded elsewhere
% Options for packages loaded elsewhere
\PassOptionsToPackage{unicode}{hyperref}
\PassOptionsToPackage{hyphens}{url}
\PassOptionsToPackage{dvipsnames,svgnames,x11names}{xcolor}
%
\documentclass[
]{article}
\usepackage{xcolor}
\usepackage[top=25mm,bottom=25mm,left=25mm,right=25mm]{geometry}
\usepackage{amsmath,amssymb}
\setcounter{secnumdepth}{5}
\usepackage{iftex}
\ifPDFTeX
  \usepackage[T1]{fontenc}
  \usepackage[utf8]{inputenc}
  \usepackage{textcomp} % provide euro and other symbols
\else % if luatex or xetex
  \usepackage{unicode-math} % this also loads fontspec
  \defaultfontfeatures{Scale=MatchLowercase}
  \defaultfontfeatures[\rmfamily]{Ligatures=TeX,Scale=1}
\fi
\usepackage{lmodern}
\ifPDFTeX\else
  % xetex/luatex font selection
\fi
% Use upquote if available, for straight quotes in verbatim environments
\IfFileExists{upquote.sty}{\usepackage{upquote}}{}
\IfFileExists{microtype.sty}{% use microtype if available
  \usepackage[]{microtype}
  \UseMicrotypeSet[protrusion]{basicmath} % disable protrusion for tt fonts
}{}
\makeatletter
\@ifundefined{KOMAClassName}{% if non-KOMA class
  \IfFileExists{parskip.sty}{%
    \usepackage{parskip}
  }{% else
    \setlength{\parindent}{0pt}
    \setlength{\parskip}{6pt plus 2pt minus 1pt}}
}{% if KOMA class
  \KOMAoptions{parskip=half}}
\makeatother
% Make \paragraph and \subparagraph free-standing
\makeatletter
\ifx\paragraph\undefined\else
  \let\oldparagraph\paragraph
  \renewcommand{\paragraph}{
    \@ifstar
      \xxxParagraphStar
      \xxxParagraphNoStar
  }
  \newcommand{\xxxParagraphStar}[1]{\oldparagraph*{#1}\mbox{}}
  \newcommand{\xxxParagraphNoStar}[1]{\oldparagraph{#1}\mbox{}}
\fi
\ifx\subparagraph\undefined\else
  \let\oldsubparagraph\subparagraph
  \renewcommand{\subparagraph}{
    \@ifstar
      \xxxSubParagraphStar
      \xxxSubParagraphNoStar
  }
  \newcommand{\xxxSubParagraphStar}[1]{\oldsubparagraph*{#1}\mbox{}}
  \newcommand{\xxxSubParagraphNoStar}[1]{\oldsubparagraph{#1}\mbox{}}
\fi
\makeatother
\usepackage{listings}
\newcommand{\passthrough}[1]{#1}
\lstset{defaultdialect=[5.3]Lua}
\lstset{defaultdialect=[x86masm]Assembler}


\usepackage{longtable,booktabs,array}
\usepackage{calc} % for calculating minipage widths
% Correct order of tables after \paragraph or \subparagraph
\usepackage{etoolbox}
\makeatletter
\patchcmd\longtable{\par}{\if@noskipsec\mbox{}\fi\par}{}{}
\makeatother
% Allow footnotes in longtable head/foot
\IfFileExists{footnotehyper.sty}{\usepackage{footnotehyper}}{\usepackage{footnote}}
\makesavenoteenv{longtable}
\usepackage{graphicx}
\makeatletter
\newsavebox\pandoc@box
\newcommand*\pandocbounded[1]{% scales image to fit in text height/width
  \sbox\pandoc@box{#1}%
  \Gscale@div\@tempa{\textheight}{\dimexpr\ht\pandoc@box+\dp\pandoc@box\relax}%
  \Gscale@div\@tempb{\linewidth}{\wd\pandoc@box}%
  \ifdim\@tempb\p@<\@tempa\p@\let\@tempa\@tempb\fi% select the smaller of both
  \ifdim\@tempa\p@<\p@\scalebox{\@tempa}{\usebox\pandoc@box}%
  \else\usebox{\pandoc@box}%
  \fi%
}
% Set default figure placement to htbp
\def\fps@figure{htbp}
\makeatother





\setlength{\emergencystretch}{3em} % prevent overfull lines

\providecommand{\tightlist}{%
  \setlength{\itemsep}{0pt}\setlength{\parskip}{0pt}}



 


\newenvironment{keep_together}{\par\noindent\begin{minipage}{\linewidth}}{\end{minipage}\par}
\usepackage{float}
\usepackage{pdflscape}
\usepackage{framed}
\usepackage{fvextra}
\usepackage{xcolor}
\newenvironment{keep_together}{\begin{minipage}{\linewidth}}{\end{minipage}}
\lstset{
  basicstyle=\ttfamily\small,
  breaklines=true,
  breakatwhitespace=false,
  frame=single,
  columns=fullflexible,
  keepspaces=true
}
\usepackage{etoolbox}
\definecolor{shadecolor}{RGB}{248,248,248}
\BeforeBeginEnvironment{verbatim}{\begin{snugshade}}
\AfterEndEnvironment{verbatim}{\end{snugshade}}
\DefineVerbatimEnvironment{Highlighting}{Verbatim}{breaklines,commandchars=\\{}}
\RecustomVerbatimEnvironment{verbatim}{Verbatim}{breaklines}
\makeatletter
\@ifpackageloaded{caption}{}{\usepackage{caption}}
\AtBeginDocument{%
\ifdefined\contentsname
  \renewcommand*\contentsname{Table of contents}
\else
  \newcommand\contentsname{Table of contents}
\fi
\ifdefined\listfigurename
  \renewcommand*\listfigurename{List of Figures}
\else
  \newcommand\listfigurename{List of Figures}
\fi
\ifdefined\listtablename
  \renewcommand*\listtablename{List of Tables}
\else
  \newcommand\listtablename{List of Tables}
\fi
\ifdefined\figurename
  \renewcommand*\figurename{Figure}
\else
  \newcommand\figurename{Figure}
\fi
\ifdefined\tablename
  \renewcommand*\tablename{Table}
\else
  \newcommand\tablename{Table}
\fi
}
\newcommand*\listoflistings\lstlistoflistings
\AtBeginDocument{%
\renewcommand*\lstlistlistingname{List of Listings}
}
\makeatother
\makeatletter
\makeatother
\makeatletter
\@ifpackageloaded{caption}{}{\usepackage{caption}}
\@ifpackageloaded{subcaption}{}{\usepackage{subcaption}}
\makeatother
\usepackage{bookmark}
\IfFileExists{xurl.sty}{\usepackage{xurl}}{} % add URL line breaks if available
\urlstyle{same}
\hypersetup{
  pdftitle={Strategic Churn Analysis \& Retention Plan},
  pdfauthor={Joshua H.},
  colorlinks=true,
  linkcolor={blue},
  filecolor={Maroon},
  citecolor={Blue},
  urlcolor={Blue},
  pdfcreator={LaTeX via pandoc}}


\title{Strategic Churn Analysis \& Retention Plan}
\usepackage{etoolbox}
\makeatletter
\providecommand{\subtitle}[1]{% add subtitle to \maketitle
  \apptocmd{\@title}{\par {\large #1 \par}}{}{}
}
\makeatother
\subtitle{Prioritized Strategy to Reduce Customer Churn}
\author{Joshua H.}
\date{26 December 2025}
\begin{document}
\maketitle

\renewcommand*\contentsname{Table of contents}
{
\hypersetup{linkcolor=}
\setcounter{tocdepth}{3}
\tableofcontents
}

\section{Executive Summary}\label{executive-summary}

We have completed a comprehensive analysis of our customer base to
identify key drivers of churn and specific opportunities for revenue
retention.

\begin{itemize}
\tightlist
\item
  \textbf{The Problem}: Our churn is predictable but costly. Our model
  indicates that high-value customers on month-to-month contracts are
  leaving at an accelerated rate.
\item
  \textbf{The Solution}: We have deployed a predictive model that
  identifies at-risk customers with \textbf{\%82.8} accuracy (AUC). By
  tuning this model to be aggressive, we can flag 75\% of potential
  churners before they leave.
\item
  \textbf{The Opportunity}: We have generated a prioritized ``Hit List''
  of 500 active customers. Intervention with this specific group
  represents a potential revenue save of roughly \$17,616.53 per month.
\end{itemize}

\section{Immediate Action: The ``Hit
List''}\label{immediate-action-the-hit-list}

Our primary recommendation is to immediately empower the Customer
Success Team to contact the top 500 at-risk customers.

The table below highlights our \textbf{``Low Hanging Fruit''}---active
customers who are currently in good standing but have a Probability of
Churn \textgreater{} 40\% and a high Monthly Bill. These are the most
expensive customers to lose.

\phantomsection\label{hit_list_table}
\phantomsection\label{hit_list_table-1}
\begin{longtable}[]{@{}llll@{}}
\toprule\noalign{}
Customer ID & Churn Risk & Monthly Bill & Expected Loss \\
\midrule\noalign{}
\endhead
\bottomrule\noalign{}
\endlastfoot
4132-KALRO & 90.0\% & \$100.85 & \$90.77 / mo \\
5630-IXDXV & 84.0\% & \$106.35 & \$89.33 / mo \\
1393-IMKZG & 92.0\% & \$95.85 & \$88.18 / mo \\
8087-LGYHQ & 93.0\% & \$94.05 & \$87.47 / mo \\
5168-MSWXT & 86.0\% & \$94.75 & \$81.48 / mo \\
\end{longtable}

\phantomsection\label{hit_list_table-2}
\begin{lstlisting}
 Customer ID Churn Risk Monthly Bill Expected Loss
\end{lstlisting}

3270 4132-KALRO 90.0\% \$100.85 \$90.77 / mo 2474 5630-IXDXV 84.0\%
\$106.35 \$89.33 / mo 2238 1393-IMKZG 92.0\% \$95.85 \$88.18 / mo 2097
8087-LGYHQ 93.0\% \$94.05 \$87.47 / mo 448 5168-MSWXT 86.0\% \$94.75
\$81.48 / mo

\subsection{Strategic Drivers: Why They
Leave}\label{strategic-drivers-why-they-leave}

Our analysis has identified the three strongest predictors of customer
churn. This provides a roadmap for systemic improvements.

\subsection{Driver \#1: Price Sensitivity (Monthly
Charges)}\label{driver-1-price-sensitivity-monthly-charges}

\begin{itemize}
\item
  \textbf{Insight}: MonthlyCharges is the \#1 predictor of churn
  (Importance: 18.8\%). Customers with bills exceeding \$80/month are
  significantly more likely to cancel.
\item
  \textbf{Implication}: Our ``Premium'' tier is a churn factory. We may
  be over-pricing or under-delivering value at this tier.
\end{itemize}

\subsection{Driver \#2: The ``Trial'' Period
(Tenure)}\label{driver-2-the-trial-period-tenure}

\begin{itemize}
\item
  \textbf{Insight}: Tenure is the \#2 predictor. The risk of churn is
  highest in months 1--6. Once a customer survives past month 6, their
  lifetime value increases dramatically.
\item
  \textbf{Implication}: Our onboarding process is critical. A ``First 90
  Days'' engagement program is necessary to bridge this gap.
\end{itemize}

\subsection{Driver \#3: Contract
Structure}\label{driver-3-contract-structure}

\begin{itemize}
\item
  \textbf{Insight}: Moving a customer from a Month-to-Month contract to
  a One-Year contract is the single most effective way to reduce their
  churn probability.
\item
  \textbf{Implication}: Month-to-month contracts offer flexibility to
  the customer but instability to the company.
\end{itemize}

\section{Proposed Pilot Programs}\label{proposed-pilot-programs}

Based on these drivers, we propose two pilot programs for Q1 2026:

\subsection{Pilot A: The ``Stickiness'' Campaign
(Sales)}\label{pilot-a-the-stickiness-campaign-sales}

\begin{itemize}
\item
  \textbf{Target}: High-risk customers on Month-to-Month contracts.
\item
  \textbf{Offer}: A \$10/month discount for 12 months in exchange for
  signing a 1-year contract.
\item
  \textbf{Goal}: Shift 20\% of the at-risk cohort to annual contracts,
  stabilizing recurring revenue.
\end{itemize}

\subsection{Pilot B: The ``Auto-Pay'' Push
(Marketing)}\label{pilot-b-the-auto-pay-push-marketing}

\begin{itemize}
\item
  \textbf{Target}: Customers paying via ``Electronic Check'' (who churn
  at nearly 2x the rate of Credit Card users).
\item
  \textbf{Offer}: A one-time \$5 bill credit for enabling Auto-Pay.
\item
  \textbf{Goal}: Reduce payment friction, a known silent killer of
  retention.
\end{itemize}

\section{Model Performance \&
Methodology}\label{model-performance-methodology}

\begin{itemize}
\item
  \textbf{Predictive Power}: The Random Forest model achieved an ROC AUC
  score of \%82.8, indicating strong ability to distinguish between
  loyal and churning customers.
\item
  \textbf{Strategy}: We utilized a Threshold Moving strategy, lowering
  the decision boundary to 30\%.

  \begin{itemize}
  \item
    \textbf{\emph{Why?}} This deliberately increases sensitivity. We
    accept a higher rate of ``false alarms'' (predicting churn where
    none happens) to ensure we miss fewer actual churners.
  \item
    \textbf{\emph{Business Logic}}: It is far less costly to offer a
    retention bonus to a happy customer than it is to lose a high-value
    customer because we were too conservative.
  \end{itemize}
\end{itemize}




\end{document}
